\appendix

\begin{center}
  {\Huge\bf Appendix}\\
  {\large\bf -For online publication only-}
\end{center}

\section{Derivation of Equation \ref{eq_UIP_RF} \label{Appendix_ReducedFormResults}}

The country $n$ risk-free bond pays off $P_n$ units of the traded good at maturity. We derive the value of the risk-free bond, $V_n$, by applying the asset pricing equation to the bond payoff: 
\begin{equation*}
  V_n = \mathbb{E}\left[M_{T} P_n
  \right],
\end{equation*}
where $M_{T}$ denotes the stochastic discount factor. The country $n$ risk-free rate (in levels), $R^f_n$, is the inverse of the price of the risk-free bond:
\begin{equation*}
  R^f_n = \frac{1}{V_n}.
\end{equation*}
Putting the previous two equations together yields the following relationship:
\begin{equation*}
  \mathbb{E}\left[ M_T P_n \right] R^f_n = 1.
\end{equation*}
As a result, the risk-free rates of countries $f$ and $h$ are related as follows:
\begin{equation*}
  \mathbb{E}\left[M_T P_f \right] R^f_f 
  = \mathbb{E}\left[M_T P_h \right] R^f_h = 1
\end{equation*} 
If the stochastic discount factor and prices are log-normal, we can perform the following calculations:
\begin{align*}
  & \mathbb{E}\left[M_{T} P_f \right] R^f_f
    = \mathbb{E}\left[M_{T} P_h \right] R^f_h \\
  \Leftrightarrow\quad
  & \mathbb{E}\left[\exp\left[ m_T + p_f + r^f_f \right]\right]
    = \mathbb{E}\left[\exp\left[ m_T + p_h + r^f_h \right]\right] \\
  \Leftrightarrow\quad
  & \mathbb{E}\left[m_T + p_f\right] + \frac{1}{2}\text{var}\left(m_T\right) +      \frac{1}{2}\text{var}\left(p_f\right) + \text{cov}\left(m_{T}, p_f\right) + r^f_f \\
  & = \mathbb{E}\left[m_{T}+ p_h\right] + \frac{1}{2}\text{var}\left(m_{T}\right) + \frac{1}{2}\text{var}\left(p_h\right) + \text{cov}\left(m_T, p_h\right) + r^f_h,
\end{align*}
We cancel out $\text{var}\left( m_T \right)$ from both sides of the previous equation.
\begin{align*}
  & \mathbb{E}\left[p_f\right]+\frac{1}{2}\text{var}\left(p_f\right) + \text{cov}\left(m_T, p_f\right) + r^f_f = \mathbb{E}\left[p_h\right]+\frac{1}{2}\text{var}\left(p_h\right)+\text{cov}\left(m_T,p_h\right) + r^f_h
  \\      \Leftrightarrow \quad
  & r^f_f+\mathbb{E}\left[p_f-p_h\right]+\frac{1}{2}\text{var}\left(p_f\right)-\frac{1}{2}\text{var}\left(p_h\right)-r^f_h = -\text{cov}\left(m_T,p_f-p_h\right)\\   
  \Leftrightarrow\quad
  &r^f_f+\log\left(\mathbb{E}\left[P_f\right]/\mathbb{E}\left[P_h\right]\right)-r^f_h = -\text{cov}\left(m_T,p_f-p_h\right)
\end{align*}
We define
$\Delta\mathbb{E}\left[s_{f,h}\right]
=\log\left(\mathbb{E}\left[P_f\right]/\mathbb{E}\left[P_h\right]\right).$
With this definition:
\begin{equation*} 
  r^f_f+\Delta\mathbb{E}\left[s_{f,h}\right]-r^f_h = -\text{cov}\left(m_T,p_f-p_h\right). 
\end{equation*}

\section{Derivation of Equation \ref{eq_BFT} \label{Appendix_BFT}}

Letting $M_n$ denote the stochastic discount factor measured in units of the consumption bundle specific to country $n$ and assuming no arbitrage, we know the risk-free rate in country $n$ satisfies:
\begin{equation*}
    R^f_n = \frac{1}{\mathbb{E} \left[ M_n\right] }
\end{equation*}
If we further assume $M_n$ is log-normally distributed, then
\begin{eqnarray*}
r^f_n &=&-\log \left( E\left[ M_n \right] \right)  \\
&=&-\left( E\left[ m_n \right] +\frac{1}{2}\text{var}\left[ m_n \right] \right) 
\end{eqnarray*}
Next, take the difference between the risk-free rates of countries $f$ and $h$:
\begin{eqnarray*}
r^f_f - r^f_h  
&=& -\left( \mathbb{E} \left[ m_f \right] +\frac{1}{2}\text{var}\left[ m_f \right] \right) 
+ \left( \mathbb{E}\left[ m_h \right] +\frac{1}{2}\text{var}\left[ m_h \right] \right)  \\
r^f_f-r^f_h + \mathbb{E} \left[ m_f - m_h \right] 
&=& \frac{1}{2}\left( \text{var}\left[ m_h \right] - \text{var}\left[ m_f \right] \right) 
\end{eqnarray*}
Finally, when financial markets are complete, $\Delta s_{f, h} = m_f - m_h$. Thus
\begin{eqnarray*}
r^f_f + \mathbb{E} \Delta s_{f, h} - r^f_h 
&=& \frac{1}{2}\left( \text{var}\left[ m_h \right] - \text{var}\left[ m_f \right] \right)
\end{eqnarray*}