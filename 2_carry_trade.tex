\section{Risk-free Interest Rates, the Carry Trade, and Failures of
  Uncovered Interest Parity}

Early work on currency risk began as an attempt to understand three
puzzling empirical facts in currency markets. The first, often referred to as the \textit{forward premium puzzle}, is that
regressions of changes in the exchange rates on differences in
interest rates yield a coefficient smaller than one, suggesting that,
on average, high-interest-rate currencies do not depreciate enough to
wipe out any differences in interest rates \citep{Bilson1981, Fama1984}.
The second is that investors in the \textit{carry trade} appear to be
making money by borrowing in currencies with low interest
rates and lending in currencies with high interest rates. The third puzzling fact is that differences in 
interest rates and currency returns are highly persistent, so that the same
countries tend to have high or low interest rates on a very long-term
basis.



\begin{textbox}[h]\section{Measuring Risk-free Interest Rates and Currency Returns}

Before we characterize these three facts in more detail, it is worth
taking a moment to think about how to measure and compare returns and risk-free
interest rates across currencies. In the United States we often regard
T-Bills as risk-free. However, even if we assume the US government
cannot default, yields on government debt of many other countries are
almost certainly contaminated with the possibility of government
default.

For this reason, most of the recent literature follows \cite{LustigRoussanovVerdelhan2011} by
constructing synthetic risk-free rates using currency forward
contracts and the Covered Interest Parity (CIP).\footnote{A forward
  contract is a rate at which a bank agrees to exchange one currency
  for another at a pre-specified date in the future.} To understand
CIP, consider again our example of Japan and New Zealand, and suppose
there is a risk-free savings account in each country. A Japanese
investor with access to both these accounts and to currency forward
contracts then has two ways of making a risk-free investment in yen.
First, she can simply save at the yen deposit rate. Alternatively, she
can convert yen into New Zealand dollars at the spot exchange rate,
invest the New Zealand dollars at the local savings rate, and sign a
forward contract to exchange her New Zealand dollars back to yen at the
end of her investment period. Both strategies are risk-free since the
spot and forward exchange rates are known today. Therefore, if there
is no arbitrage, the two strategies must yield the same return:
\begin{equation}
  \underbrace{1 + r_{JPY}}_{\text{yen risk-free rate}}
  = \underbrace{
    (1 + r_{NZD}) \times \frac{S}{F}
  }_{\text{Implied yen risk-free rate}}.
\end{equation}
Taking logs on both sides of this equation and rearranging shows
$r_{JPY}-r_{NZD}=s-f$ --- The difference in risk-free interest
rates must equal the difference between the (log) spot and forward
exchange rates. This difference is known as the forward premium. As a
result, we can reliably measure risk-free interest rate differentials
from forward prices, even if a given government's ability to repay its
debts may be in question.\footnote{For this reason, commercial providers of the
carry trade also tend to implement it by buying and selling forward
contracts, rather than by trading corporate or government bonds.}.\end{textbox}


All three of these facts have in common that they describe different violations of uncovered interest parity (UIP), that is, for a given pair of currencies $h$ and $f$, the expected rate of depreciation is usually not equal to the interest differential, $$E\Delta s \neq r^f_h-r^f_f .$$ An extensive literature has documents these violations of UIP and a number of related trading strategies. Hodrick (1987), Froot and Thaler (1990), Engel (1996), Lewis (2011), and Engel (2014) provide excellent surveys. 

One difficulty for students of this literature is that it evolved simultaneously as an attempt to understand exchange rate dynamics (e.g. the forward premium puzzle) and asset pricing anomalies (e.g. the carry trade), and thus often alternates between regression-based and portfolio-methods that are not necessarily directly comparable to each other. We next give a brief overview using the unified approach by \citet{HassanMano2015}. This allows us to show the main empirical facts in a form where regressions map one-for-one into portfolios, while also abstracting from many of the details uncovered in the two branches of the literature.

Thus, one way to implement a carry trade is to calculate risk-free 
rates for each country $i$ at the start of each month $t$ during an 
investment period $t = 1, ..., T$, and then form a portfolio where 
each currency is weighted by the difference between its risk-free rate 
and the average risk-free rate of all currencies. The return on this 
portfolio is:
\begin{equation}
  \label{eq_carry}
  \textstyle\sum_{i}\sum_t\left[ rx_{i,t+1}\left( r^f_{it}-\overline{r}^f_{t}\right) \right] ,
\end{equation}%
where, $rx_{i,t+1}=r^f_i-r^f_{USD}+\Delta s$ is the return to 
borrowing in the home currency and lending in foreign currency $i$ at
between $t$ and $t+1$.

\citet{HassanMano2015} implement this strategy for 39 currencies between 
1986 and 2010 and obtain an annualized mean return of 5.45 percent with 
a Sharpe Ratio of 0.69. For comparison, the Sharpe ratio of the U.S. 
equity market during the same time period is XX. In this sense, the carry 
trade is a profitable strategy that demands an explanation: are carry 
traders being compensated for taking on risk? If so, what kind of 
risk are they taking?

We get the third fact mentioned above by trying to understand whether we
obtaining the carry trade returns as a result of re-sorting currencies in 
our portfolio by their risk-free rates in each month. 
\citet{HassanMano2015} show that the incremental gain to re-sorting the 
carry trade portfolio every month is usually not statistically 
distinguishable from zero. Instead, the authors find that their carry 
trade strategy would have obtained 70\% of the same returns if they had 
sorted their portfolio just once at the beginning of their investment 
period using an initial three years of data, and then never changed their
portfolio afterwards.

The carry trade return in (\ref{eq_carry}) can be interpreted as the 
covariance between currency returns at $t+1$ and interest differentials 
at time $t$. \citet{HassanMano2015} therefore estimated the returns to 
the carry trade by running a predictive regression of the form:
\begin{equation}
  rx_{i,t+1}-\overline{rx}_{t+1}=\beta^{ct}\left(
    r^f_{it}-\overline{r}^f_{t}\right) +\epsilon
  _{i,t+1}^{ct}. 
  \label{eq_ct}
\end{equation} 
In the sample of 39 currencies, $\hat{\beta}^{ct}=0.67 (s.e.=0.16)$, 
implying currencies with high interest rates --- relative to other 
currencies in the same month --- on average pay significantly higher 
returns.


% where $\beta ^{ct}$ can be interpreted as the elasticity of currency
% risk premia with respect to interest rate differentials conditional on
% a time fixed effect. 


The results from estimating equation \eqref{eq_ct} also imply that for 
every dollar carry traders make on interest differentials, 1-0.67=0.33 
dollars are wiped out by predictable depreciations. In other words,
high-interest-rate currencies depreciate on average, but not by 
enough to eliminate the interest differential. We should note that 
there has been some confusion in the literature on this final point. 
In many older papers, authors show regressions of exchange rates on 
interest rate differentials with a country-specific intercept:
\begin{equation}
    \Delta s_{i,t+1} 
    = \alpha_i + \gamma \left(r^f_{i, t} - r^f_{US, t}\right) + \nu_{i, t+1},
\label{eq_fama} 
\end{equation}
and interpret the fact that the slope coefficient $\gamma$ is often 
negative as evidence that investors expect high-interest-rate 
currencies to appreciate instead of depreciate. This finding has 
prompted a considerable theoretical literature trying to rationalize 
extremely volatile currency risk premia that are negatively correlated 
with predicted depreciations (the ``Fama conditions''). However, this
interpretation is incorrect, because a regression with currency fixed
effects, such as equation \eqref{eq_fama} is not predictive -- investors at
time $t$ do not know what each currency's fixed effect is. Once one
corrects for this discrepancy between realized and predicted
appreciations, one recovers results consistent withe the view that, on
average, investors expect high-interest-rate currencies to depreciate,
not appreciate.

To summarize, the carry trade is highly profitable because there are 
highly persistent differences in risk-free interest rates across 
currencies that are only partially reversed by predictable 
depreciations. Currencies with higher interest rates on average pay 
higher returns to international investors than currencies with lower 
interest rates. In addition, there is some evidence that expected 
currency returns move over time with variation in interest rates, 
so that currencies with unusually high interest rates also pay 
unusually high returns.\footnote{In addition to the three puzzling
facts discussed in this section, there are a number of empirical 
puzzles that we do not have the space to discuss in detail. 
\citet{LRV2014}, \citet{ChinnMeredith2004}, \citet{LustigStathopoulosVerdelhan2019}
\citet{}

Engel AER, Valchev JMP, 
Lustig and richmond some currencies appreciate when they have 
high interest rates}

Taken together, [questions that organize the next sections]

