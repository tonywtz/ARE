
\subsection{Theory: Microfoundations of Safe Haven Currencies}

Prompted by this empirical evidence, the theoretical literature has
identified several fundamental economic forces that may make a given
currency safer or riskier from the perspective of global investors.
Although there is an ongoing debate on which of these forces may be
most important in practice, many of of these microfounded models share
a common structure, which we can summarize using just a few key
equations.

Consider a world economy in which international assets are priced by
the marginal utility of an international investor, $\lambda_T$, which
will be our measure of ``good'' and ``bad'' times. Times are good when
$\lambda_T$ is low, and times are bad when it is high. In different
models, $\lambda_T$ may be the marginal utility of consumption of
traded goods (the part of consumption that his shared
internationally), the marginal utility of consumption of a key
investor who intermediates between segmented international markets, or
the capital constraint of such an intermediary. Households consume a
country-specific final good, the price of which (accounted for in a
common unit) depends on $\lambda_T$ and a country-specific shock,
$x^n$,
\begin{equation}
  p^{n} = a\lambda_{T} + b x^{n}.  
  \label{eq_RF}
\end{equation}%
The country-specific shock interchangeably may stem from a
country-specific supply, demand, or monetary shock; in other words, it
is a stand-in for any factor that affects the price of consumption in
one country more than in others. The higher $x^{n}$, the higher the
price of domestic consumption relative to that in other countries. For
simplicity, assume $\lambda _{T}\sim N(0,\sigma^2_{\lambda_{T}})$ and
$x^{n} \sim N(0,\sigma^2_x) $ are normally distributed, not
necessarily independent, shocks and $a$ and $b$ are positive constants.

The real exchange rate, $S^{f, h}$ between two countries $f$ and $h$ 
is the relative price of their respective final goods. In logs,
\begin{equation}
  s^{f,h} 
  = p^f - p^h 
  = b(x^f - x^h),
\label{eq_RER}
\end{equation}
where the second equality substitutes Equation \eqref{eq_RF}. Because 
of $x^{n}$, the relative price of consumption may differ between the 
two countries, allowing the real exchange rate to move. When the price of
consumption in country $f$ increases, its consumption bundle
appreciates relative to country $h$. In this sense, we can speak of
``currencies'' even without formally introducing money into the model.

By definition, the risk-free bond in country $h$ pays $p^h$ with
certainty, while that in country $f$ pays $p^f$ with certainty, so
that each country's risk-free bond pays exactly one unit of that
country's consumption bundle. Importantly, because the real exchange
rate may move around, country $h$'s risk-free bond is not risk-free
from the perspective of households in country $f$, and vice versa. For
this reason, one country's risk-free bond may be more expensive than
the other country's risk-free bond. The risk-free rate of interest is
the yield to maturity of the risk-free bond, so that the country with
the more expensive risk-free bond automatically must have the lower
risk-free rate of interest. In this sense, any model with a variable
real exchange rate as in equation (\ref{eq_RER}) may produce
differences in risk-free interest rates across countries.

Different theories of such differences in interest rates apply
elementary asset pricing to equation (\ref{eq_RER}). One can show that
the log expected return to borrowing in country $h$ and to lending in
country $f$ is
\begin{equation}
  r^{f} + \Delta \mathbb{E} s^{f,h} - r^{h} =cov\left( \lambda _{T},p^{h}-p^{f}\right),
  \label{eq_UIP_RF}
\end{equation}%
where $r^{n}$ is the risk-free interest rate in country
$n$.\footnote{$\Delta\mathbb{E}s^{f,h}$ is defined as the logarithm of
  the ratio of the countries' expected real price changes. See
  Appendix \ref{Appendix_ReducedFormResults} for a formal derivation.}
This statement means a currency that tends to appreciate when
$\lambda_T$ is high pays a lower expected return and, if
$\Delta \mathbb{E} s^{f,h}\approx0$ (as is the case in the data),
therefore must has a lower risk-free interest rate. That is, a
currency that appreciates in bad times (for example, in times when
consumption goods are expensive everywhere) provides a hedge against
worldwide consumption risk and pays lower returns in equilibrium.

Equations \eqref{eq_RF} and \eqref{eq_UIP_RF} are the main ingredients
of risk-based models of unconditional differences in interest rates
across countries, where different approaches have identified different
reasons why the countries differ in the degree to which their price
indices covary with $\lambda_T$.

\paragraph*{Country Size}




FIXMETH Describe
\begin{enumerate}
\item \citet{Hassan2013}

  \begin{equation} \lambda_{T}^\ast = -c \sum_{n} \theta^n x^n,
    \label{eqn:lambdat2NP}
  \end{equation}
  larger countries have lower interst rates stock returns traded
  nontraded sector Currency Union

\begin{equation}
  r^{f \ast} + \Delta \mathbb{E} s^{f, h \ast} - r^{h \ast}
  =g\left(\theta^h - \theta^f\right) \sigma_N^2.
  \label{eq_FF_UIP}
\end{equation}
\item \citet{Martin2012}
\item \citet{Richmond2019}

$$\lambda_{T}^\ast = -c
\sum_{n} \theta^n x^n- d\sum_{n} \nu^n x^n$$


\item \citet{Readyetal2013} (and Maggiori JMP?)
$$p^{\text{Commodity Country}}=a\lambda_T+b\frac{\text{Finished goods shipped}}{\text{Shipping Capacity}} $$
Commodity producers appreciate in good times and depreciate in bad
times Their exchange rates are positively correlated with the
commodity price and shipping costs.
\item \citet{FarhiGabaix2016} Countries differ in their resilience to
  disaster risk. Conditional on a disaster striking the world economy,
  the currencies of more resilient countries appreciate.
\item \citet{GourinchasGovillotRey2017} Differences in country size and
  risk aversion
\item Della Corte, Riddiough and Sarno (2016)
\item Tran (2013)
\item Powers (2015)
\item Wiriadinata (2020)
\end{enumerate}

\subsection{Evidence}

[Puzzle 1: microfoundation which is it]

If there is indeed something to this story, it changes how we think
about central issues in macro.
