\begin{abstract}
  This article reviews the literature on currency and country risk with a focus on its macroeconomic origins and implications. A growing body of evidence shows countries with safer currencies enjoy persistently lower interest rates and a lower required return to capital. As a result, they accumulate relatively more capital than countries with currencies international investors perceive as risky. Whereas earlier research focused mainly on the role of currency risk in generating violations of uncovered interest parity and other financial anomalies, more recent evidence points to important implications for the allocation of capital across countries, the efficacy of exchange rate stabilization policies, the sustainability of trade deficits, and the spillovers of shocks across international borders.
\end{abstract}
