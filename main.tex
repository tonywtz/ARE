%&pdflatex

\documentclass[12pt,letter]{article}
%%%%%%%%%%%%%%%%%%%%%%%%%%%%%%%%%%%%%%%%%%%%%%%%%%%%%%%%%%%%%%%%%%%%%%%%%%%%%%%%
\usepackage{booktabs, caption, subcaption, float}
\usepackage[latin1]{inputenc}
\usepackage{longtable}
\usepackage{graphics}
\usepackage{geometry}
\usepackage{graphicx}
\usepackage{amsfonts, amssymb, amsmath}
\usepackage{natbib, multibib}
\usepackage{theorem}
\usepackage{setspace}
\usepackage[normalem]{ulem}
\usepackage{caption}
\usepackage{tabularx}
\usepackage{epstopdf}
\usepackage{rotating}
\usepackage{placeins}
\usepackage[makeroom]{cancel}
\usepackage{varioref}
\usepackage{hyperref}

\setcounter{MaxMatrixCols}{10}
% TCIDATA{OutputFilter=Latex.dll} TCIDATA{Version=5.50.0.2960}
% TCIDATA{<META NAME="SaveForMode" CONTENT="1">}
% TCIDATA{BibliographyScheme=BibTeX} TCIDATA{LastRevised=Monday, May
% 11, 2015 19:08:26} TCIDATA{<META NAME="GraphicsSave" CONTENT="32">}

\theoremstyle{break} \theorembodyfont{\normalfont\itshape}
\newtheorem{thm}{Theorem}
\theoremstyle{break}
\theorembodyfont{\normalfont\itshape}
\newtheorem{corrollary}{Corollary} \theoremstyle{break}
\theorembodyfont{\normalfont\itshape} \newtheorem{prop}{Proposition}
\theoremstyle{break} \theorembodyfont{\normalfont\itshape}
\newtheorem{lemma}{Lemma} \theoremstyle{break}
\theorembodyfont{\normalfont\itshape} \newtheorem{algm}{Algorithm}
\theoremstyle{break} \theorembodyfont{\normalfont\itshape}
\newtheorem{definition}{Definition} \theoremstyle{break}
\theorembodyfont{\normalfont\itshape}
\newtheorem{condition}{Condition} \theoremstyle{break}
\theorembodyfont{\normalfont\itshape}

\renewcommand{\baselinestretch}{1.5}
\setlength{\hoffset}{-0.4in}
\setlength{\textwidth}{6.8in}
\setlength{\textheight}{9.3in}
\setlength{\topmargin}{-0.5in}



\begin{document}

\author{Tarek A. Hassan\thanks{\textbf{Boston University}, NBER, and
    CEPR; Postal Address: 270 Bay State Road, Boston, MA 02215, USA;
    E-mail: thassan@bu.edu.} \and  Tony
  Zhang\thanks{\textbf{%
      Boston University, Questrom School of Business}; Postal Address:
    595 Commonwealth Avenue, Boston, MA 02215, USA; E-mail:
    tzhang0@bu.edu.} } \title{The Economics of Currency Risk Premia\thanks{}}
\date{ \bigskip July 2020}
\maketitle

% +Title

\thispagestyle{empty}


\vspace{-0.7cm}


\setstretch{1.0}



\begin{abstract}
This article reviews the literature on currency risk with a focus on its macroeconomic origins and implications. A growing body of evidence shows that countries with safer currencies enjoy persistently lower interest rates, a lower required return to capital, and accumulate relatively more capital than countries international investors perceive as risky. While earlier research has focused mainly on the role of currency risk in generating violations of uncovered interest parity and other financial anomalies, more recent evidence also points to important implications for the allocation of capital across countries, the efficacy of exchange rate stabilization policies, the sustainability of trade deficits, and the spill-over of shocks across borders.
\end{abstract}


\bigskip

\bigskip {\noindent \textbf{JEL classification:} }

{\noindent \textbf{Keywords:}  }

\pagebreak

\setstretch{1.4} \setcounter{page}{1}


\section{Introduction}


A key tenet in economics is that the degree to which firms should be willing to invest in a given project depends crucially on the required rate of return to capital: a price-taking firm should install just enough capital so that the marginal product of capital, $MPK_i$ equals the required rate of return to capital,$r_i$, \begin{equation}MPK_i=\underbrace{r^f+RP_i}_{ r_i}.\label{eq_one}\end{equation} 
This equation is the point of departure for many classic questions in economics. Students of asset pricing are taught that a firm's $r_i$ has two components; a risk-free part, $r^f$, and a risk-premium, $RP_i$, that depends on the firm's risk characteristics. One of the classic puzzles in asset pricing is why $RP_i$ is so large relative to $r^f$ (the equity premium puzzle). Monetary economists are interested in the Federal Reserve's power to manipulate $r^f$, while students of economic growth often take differences in the $MPK_i$ as a measure of inefficiencies in the allocation of capital across countries.

In this article we argue that recent insights from the study of currency risk premia have important lessons for how we should think about equation (\ref{eq_one}), and by extension, for its key applications in asset pricing, macroeconomics, and economic growth. The simplest, and perhaps and most important, of these lessons is that countries differ dramatically in their risk-free interest rates. These differences in risk-free interest rates are large (on the same order of magnitude as the equity premium puzzle), appear to be highly persistent over time (lasting for decades rather than years), and cannot be explained by predictable depreciations, government default, or differences in inflation rates. Instead, these differences in interest rates appear intimately linked to the risk characteristics of the country's exchange rate, and to its exchange rate regime.   

Figure XX shows the difference in the risk-free interest rates of the New Zealand Dollar and the Japanese Yen as an example (we will discuss below how one can measure such risk-free rates and compare them across countries). The Figure shows that the New Zealand dollar had a higher risk-free rate than the Japanese Yen in \textit{every} month between 19XX and 20XX. On average this difference was about 6pp on an annualized basis. When we adjust for movements in exchange rates over the period, the difference in returns on the two countries currencies is XXpp --  meaning that a US investors who borrowed in Yen and lent in New Zealand Dollars on average made a return of XX percent over this period. (For comparison, the equity premium on US stocks was about XXpp during the same period.)

We now know that such highly persistent differences in interest rates, similar to those between New Zealand and Japan, are common in the data, even among developed economies. These large differences in interest rates across countries do not appear to be equalizing over time and translate into large differences in returns investors can earn when investing in these currencies. 

Why would $r^f$ differ permanently across countries? The emerging consensus in the literature is that the most likely explanation are currency risk premia -- the idea that some currencies are safer investments than others. A useful way of thinking about this problem is to take the perspective of a retail investor in a third country. Banks in many countries offer savings accounts denominated in in multiple currencies. 

As it turns out, this simple intuition has a lot of support in the data. For example, a seminal paper by Lustig and Verdelhan (2007) shows that currencies with low interest rates on tend to appreciate when US consumption growth is low, and depreciate when US consumption growth is high. That is, there is direct evidence that currencies with low interest rates appreciate when times are ``bad'', making them a safer store of value for US investors.

In addition to this empirical evidence, the theoretical work on currency risk has identified compelling theoretical reasons to expect the emergence of safe-haven currencies and long-lasting differences in interest rates. In a nutshell, currency risk premia arise naturally in a wide range of international macro models. For example, Hassan (2013) shows that simply allowing for some economies to be larger than others within a standard international real business cycle model is sufficient to generate long-term differences in interest rates between countries, because the currencies of larger countries tend to appreciate when world-wide output is low. That is, even within the most canonical, frictionless, international macro models, currency risk premia tend to arise naturally in equilibrium. Other authors have similarly pointed to the emergence of currency premia in models with intermediary capital constraints, trade costs, and differences in resource endowments, among others. 

A major difficulty this literature shares with a broader literature in asset pricing is that models with conventional preferences tend to produce risk premia that are quantitatively small. That is, although a number of papers have identified compelling reasons why, for example, the interest rate in Japan should be lower than that in New Zealand, most of these models suggest that it should be lower by something on the order of 0.06pp rather than the 6pp we measure in the data. In this sense, the literature is running into a quantitative ``interest differential puzzle,'' which is in some ways analogous to the equity premium puzzle. Both puzzles fundamentally struggle with the prediction of models with standard preferences that risk premia should be small given the relatively small aggregate variation in consumption growth we measure in the data. Quantitative research on currency risk is thus a major area for future research.

We survey the rapidly growing empirical and theoretical literatures on currency risk premia in detail in sections XX and XX. Because the initial focus of this literature was mainly on resolving asset pricing anomalies, many of its key papers tend to use the language of finance. For this reason we attempted to keep this review as non-technical as possible, focusing as much as possible on the underlying economics.

Although the literature on currency risk premia has proliferated in recent years, it has perhaps been less successful at making clear the relevance of these premia beyond a financial context, a gap we hope to partially fill with this article. 

Perhaps the most immediate implication of currency premia for the real economy is for capital accumulation:  If countries differ persistently in $r^f$, then those with hither interest rates have a persistently higher cost of capital and thus, according to (\ref{eq_one}) should produce with relatively less capital. Returning to our example from Figure XX, it turns out that indeed the capital-to-output ratio K/Y in New Zealand is XX percent lower than in Japan, suggesting that, indeed, the marginal product of capital is larger in New Zealand than it is in Japan. More generally, countries with persistently higher interest rates appear to have higher marginal products of capital in the long-run. This simple insight has direct implications for several strands of the macroeconomic literature. 

The first is for the so-called Lucas-puzzle (Lucas, 1988), which posits that 


If there is indeed something to this story, it changes how we think
about central issues in macro.

1) Capital allocation
2) Efficiency calculations
3) Balance sheets of firms in emerging markets
4) Exchange rate regimes and risk
5) Dynamics, capital flows and risk


\section{Risk-free Interest Rates, Exchange Rates, and Currency returns}

Measuring Risk-free interest rates, show data

Hassan and Mano facts, Violations of UIP, Fama 84



\section{Interest Rate Differentials are Differences in Risk Premia}
\subsection{Theory}
\begin{enumerate}
    \item Bare-bones model, no differences in r when risk-neutral
    \item Get differences in r when risk averse
    \item Pretty much any asymmetry you put into a standard macro model will produce differences in r
\end{enumerate}

\subsection{Evidence that currency returns depend on risk}
\begin{enumerate}
\item[-] Some currencies are safer than others
\item SIMPLE PICTURE
\item[-] Risk factors: People in finance have shown this formally
\item Brunnermeier, Nagel and Pedersen (2008)
\end{enumerate}

Continuing with our example from Figure XX, Figure XX plots the New Zealand dollar - Japanese yen exchange rate in terms of dollars per yen. An increase in the exchange rate indicates yen appreciation. Figure XX highlights three distinct period of Japanese yen appreciation, which correspond to three periods of global economic turmoil: The early 1990s recession (1990 - 1994), the Asian financial crisis (1997 - 1998), and the Great Recession (2007 - 2009). As a result of the yen's tendency to appreciate during periods of economic turmoil, investors should naturally consider the Japanese yen a safer currency. Hence, the lower returns earned on investing in the Japanese yen may be a result of the yen's usefulness as a hedge against bad times. On the other hand, investing in the New Zealand dollar should earn higher returns because the New Zealand dollar is a riskier currency that provides none of the hedging benefits of a yen investment.

In a seminal paper, Lustig and Verdelahn (2007) provides systematic evidence that the currencies paying low excess returns to the U.S. investor indeed tend compensate the U.S. investor by providing a hedge against U.S. consumption growth risk. Their main innovation is to sort currencies into portfolios based on the currency's interest rate, and then explain the returns of these portfolios rather than the returns of the currencies themselves. 

This procedure of sorting financial assets into portfolios and then explaining the returns of portfolios of assets has a long history in empirical asset pricing. Fama French (XXXX). Intuitively, averaging the returns of currencies within portfolios should eliminate the diversifiable, currency-specific components of risk that are unrelated to differences in interest rates. The remaining variation in returns across portfolios should thus better capture risk-return trade-off from investing in high interest rate currencies.

Menkhoff, Sarno, Schmeling, and Schrimpf (2012) show exposure to exchange rate volatility explains cross-sectional differences in currency returns: High interest rate currencies depreciate when global exchange rate volatility increases. Currencies sorted into 5 portfolios based on interest rates. Low interest rate currencies appreciate when the market is volatile.

Campbell, Meideiros and Viceira (2010) and Lustig, Roussanov and Verdelhan (2011) show global equity volatility explains cross-sectional differences in currency returns.  


\subsection{Microfoundations}
These differences in risk premia arise naturally in macro models.
\begin{enumerate}
\item Hassan (2013)
\item Gabaix and Maggiori (2015)
\item Ready Roussanov Ward (2017)
\item Richmond (2019)
\item Farhi and Gabaix (2016)
\item Della Corte, Riddiough and Sarno (2016)
\item Tran (2013)
\item Powers (2015)
\item Wiriadinata (2020)
\end{enumerate}



If there is indeed something to this story, it changes how we think
about central issues in macro.

\section{Currency Risk and the allocation of capital across countries}
There are large differences in capital accumulation across countries:
\begin{enumerate}
\item Lucas (1988)
\item Caselli and Feyrer (2007)
\item Monge-Naranjo, Sanchez and Santaeualia-Llopis (2018)
\end{enumerate}
These differences may be explained by risk premia:
\begin{enumerate}
\item di Giovanni, Kalemli-Ozcan, Ulu, Baskaya (2019)
\item Richers (2020)
\item Hassan, Mertens and Zhang (2016)
\item David, Henriksen and Simanovska
\end{enumerate}
[picture]
\section{Currency Risk and Exchange Rate Regimes}
Q: Are there other papers that argue policies can manipulate risk premia?

\section{Dynamics}
\subsection{Quantitative literature}
\begin{enumerate}
\item Colacito and Croce (2011)
\item Gourio, Siemer and Verdelhan (2013)
\item Colacito, Croce, Ho and Howard (2018)
\end{enumerate}
\subsection{Theories written for FPP}
Heyerdahl-Larsen
Statopoulous
Verdelhan
Backus Foresi Telmer
\subsection{Measurement of risk}
Maybe measurement of risk
- implied vol
- commercial indices
- newspapers
- conference call transcripts
\subsection{Global financial cycle}

\section{Capital flows and the allocation puzzle}
Capital flows run counter to neo-classical model:
\begin{enumerate}
\item Gourinchas and Jeanne (2013)
\item[-] Current explanations:
\item Differences in financial development (Ju and Wei, 2006;
  Caballero, Farhi and Gouinchas, 2008; Mendoza, Quadrini and
  Rios-Rull, 2009)
\item Differences in factor utilization (Jin, 2012)
\item Differences in contracting frictions (Aguiar and Amador, 2010)
\end{enumerate}

\section{Dynamics of Currency Risk Premia}


\section{Unsorted}
\begin{enumerate}
\item Gourinchas and Rey (2007)
\item Govillot, Rey and Gourinchas (2010)
\item[-] Implications for equilibrium portfolios and global imbalances
\item Miranda-Agrippino and Rey (2020)
\item[-] Global financial cycles
\item Boccola and Lorenzoni (forthcoming)
\item[-] Debt crises
\item Du and Schreger
\item[-] Corporate balance sheets
\item[-] Special role for the dollar
\item Hassan, Mertens and Zhang (2020)
\item Burnside, Eichenbaum, Kleshchelski and Rebelo (2011)
\end{enumerate}



\end{document}