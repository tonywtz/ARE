%&pdflatex

\documentclass[12pt,letter]{article}
%%%%%%%%%%%%%%%%%%%%%%%%%%%%%%%%%%%%%%%%%%%%%%%%%%%%%%%%%%%%%%%%%%%%%%%%%%%%%%%%
\usepackage{booktabs, caption, subcaption, float}
\usepackage[latin1]{inputenc}
\usepackage{longtable}
\usepackage{graphics}
\usepackage{geometry}
\usepackage{graphicx}
\usepackage{amsfonts, amssymb, amsmath}
\usepackage{natbib, multibib}
\usepackage{theorem}
\usepackage{setspace}
\usepackage[normalem]{ulem}
\usepackage{caption}
\usepackage{tabularx}
\usepackage{epstopdf}
\usepackage{rotating}
\usepackage{placeins}
\usepackage[makeroom]{cancel}
\usepackage{varioref}
\usepackage{hyperref}

\setcounter{MaxMatrixCols}{10}
% TCIDATA{OutputFilter=Latex.dll} TCIDATA{Version=5.50.0.2960}
% TCIDATA{<META NAME="SaveForMode" CONTENT="1">}
% TCIDATA{BibliographyScheme=BibTeX} TCIDATA{LastRevised=Monday, May
% 11, 2015 19:08:26} TCIDATA{<META NAME="GraphicsSave" CONTENT="32">}

\theoremstyle{break} \theorembodyfont{\normalfont\itshape}
\newtheorem{thm}{Theorem}
\theoremstyle{break}
\theorembodyfont{\normalfont\itshape}
\newtheorem{corrollary}{Corollary} \theoremstyle{break}
\theorembodyfont{\normalfont\itshape} \newtheorem{prop}{Proposition}
\theoremstyle{break} \theorembodyfont{\normalfont\itshape}
\newtheorem{lemma}{Lemma} \theoremstyle{break}
\theorembodyfont{\normalfont\itshape} \newtheorem{algm}{Algorithm}
\theoremstyle{break} \theorembodyfont{\normalfont\itshape}
\newtheorem{definition}{Definition} \theoremstyle{break}
\theorembodyfont{\normalfont\itshape}
\newtheorem{condition}{Condition} \theoremstyle{break}
\theorembodyfont{\normalfont\itshape}

\renewcommand{\baselinestretch}{1.5}
\setlength{\hoffset}{-0.4in}
\setlength{\textwidth}{6.8in}
\setlength{\textheight}{9.3in}
\setlength{\topmargin}{-0.5in}



\begin{document}

\author{Tarek A. Hassan\thanks{\textbf{Boston University}, NBER, and
    CEPR; Postal Address: 270 Bay State Road, Boston, MA 02215, USA;
    E-mail: thassan@bu.edu.} \and  Tony
  Zhang\thanks{\textbf{%
      Boston University, Questrom School of Business}; Postal Address:
    595 Commonwealth Avenue, Boston, MA 02215, USA; E-mail:
    tzhang0@bu.edu.} } \title{The Economics of Currency Risk Premia\thanks{}}
\date{ \bigskip July 2020}
\maketitle

% +Title

\thispagestyle{empty}


\vspace{-0.7cm}


\setstretch{1.0}



\begin{abstract}
This article reviews the literature on currency risk with a focus on its macroeconomic origins and implications. A growing body of evidence shows that countries with safer currencies enjoy persistently lower interest rates, a lower required return to capital, and accumulate relatively more capital than countries international investors perceive as risky. While earlier research has focused mainly on the role of currency risk in generating violations of uncovered interest parity and other financial anomalies, more recent evidence also points to important implications for the allocation of capital across countries, the efficacy of exchange rate stabilization policies, the sustainability of trade deficits, and the spill-over of shocks across borders.
\end{abstract}


\bigskip

\bigskip {\noindent \textbf{JEL classification:} }

{\noindent \textbf{Keywords:}  }

\pagebreak

\setstretch{1.4} \setcounter{page}{1}


\section{Introduction}


A key tenet in economics is that the degree to which firms should be willing to invest in a given project depends crucially on the required rate of return to capital: a price-taking firm should install just enough capital so that the marginal product of capital, $MPK_i$ equals the required rate of return to capital,$r_i$, \begin{equation}MPK_i=\underbrace{r^f+RP_i}_{ r_i}.\label{eq_one}\end{equation} 
This equation is the point of departure for many classic questions in economics. Students of asset pricing are taught that a firm's $r_i$ has two components; a risk-free part, $r^f$, and a risk-premium, $RP_i$, that depends on the firm's risk characteristics. One of the classic puzzles in asset pricing is why $RP_i$ is so large relative to $r^f$ (the equity premium puzzle). Monetary economists are interested in the Federal Reserve's power to manipulate $r^f$, while students of economic growth often take differences in the $MPK_i$ as a measure of inefficiencies in the allocation of capital across countries.

In this article we will argue that recent insights from the study of currency risk premia have important lessons for how we should think about equation (\ref{eq_one}), and by extension, for its key applications in asset pricing, macroeconomics, and economic growth. The simplest, and perhaps and most important, of these lessons is that countries differ dramatically in their risk-free interest rates. These differences in risk-free interest rates are large (on the same order of magnitude as the equity premium puzzle), appear to be highly persistent over time (lasting for decades rather than years), and cannot be explained by predictable depreciations, government default, or differences in inflation rates. Instead, these differences in interest rates appear intimately linked to the risk characteristics of the country's exchange rate, and by extension, to its exchange rate regime.   




\section{Interest Rate Differentials are Differences in Risk Premia}
\subsection{Theory}
\begin{enumerate}
    \item Bare-bones model, no differences in r when risk-neutral
    \item Get differences in r when risk averse
\end{enumerate}

\subsection{Evidence that currency returns depend on risk}
\begin{enumerate}
\item[-] Some currencies are safer than others
\item SIMPLE PICTURE
\item Campbell, Meideiros and Viceira (2010)
\item[-] Risk factors: People in finance have shown this formally
\item Lustig and Verdelhan (2007)
\item Lustig, Roussanov and Verdelhan (2011)
\item Menkhoff, Sarno, Schmeling, and Schrimpf (2012)
\end{enumerate}

These differences in risk premia arise naturally in macro models (Microfoundations)
\begin{enumerate}
    \item Farhi and Gabaix (..)    
    \item Hassan (2013)
    \item Gabaix and Maggiori
     \item Richmond (2019)
    \item Ready Roussanov Ward (2017)  
    \item Della Corte, Riddiough and Sarno (2016)
    \item exchange rate policy HMZ (2020)
    \item Tran (2013)
    \item Powers (2015)
    \item Wiriadinata (2020)
\end{enumerate}

Quantiative literature
    \begin{enumerate}
        \item Colacito and Croce (2011)
        \item Colacito, Croce, Ho and Howard (2018)
    \end{enumerate}


If there is indeed something to this story, it changes how we think about central issues in macro

Allocation of capital across countries
\begin{enumerate}
    \item di Giovanni, Kalemli-Ozcan, Ulu, Baskaya (2019)
    \item Richers (2020)
    \item Hassan Mertens Zhang (2016)
    \item David, Henriksen and Simanovska
\end{enumerate}
Lucas (1988)
Caselli and Feyrer (2007)
Monge-Naranjo, Sanchez and Santaeualia-Llopis (2018)

Exchange rate policy
HMZ (2020)

FUTURE Directions 

External imbalances
\begin{enumerate}
    \item Gourinchas and Rey (2007)
    \item equibrium portfolios
\end{enumerate}

Special role of the dollar

Debt crises
boccola and Lorenzoni
Firm balance sheets
Du and Schreger ..


Measurement of risk



Dynamics capital flows
Rey
Global financial cycle





\section{Implications for Macro Puzzles}

\subsection{Cross-sectional Differences in K / Y}
\begin{enumerate}
    \item a
\end{enumerate}
Some statement of K/Y Lucas puzzle using this Hall and Jorgensen equation
Basic facts:
\begin{enumerate}
    \item K/Y differences strongly correlated with differences in interest rates.
    \item Are differences in interest rates to blame for K/Y puzzle?
    \item Finance people have been working on this problem and calling it the carry trade.
    \item Interest rate differentials are large and persistent  Hassan and Mano (2019)
\end{enumerate}
In this paper, we summarize the parts of this finance research that is directly important for the K/Y puzzle.



Financial effects of violation of UIP
"over borrowing"
.



\section{Future Research}
\begin{enumerate}
    \item Exchange rate policy should also affect capital accumulation (HMZ)
    \item Quantitative models / quantitative challenge for finance models

    \item Exhobitant privilege (?)
    \begin{enumerate}
        \item Govillot, Rey and Gourinchas (2010)
    \end{enumerate}
    \item Manipulated exchange rates / policy
    \item Dynamics and the allocation puzzle 
    \begin{enumerate}
        \item Gourinchas and Jeanne (2013)
        \item  Differences in financial development (Ju and Wei, 2006; Caballero, Farhi and Gouinchas, 2008; Mendoza, Quadrini and Rios-Rull, 2009)
        \item Differences in factor utilization (Jin, 2012)
        \item Differences in contracting frictions (Aguiar and Amador, 2010)
    \end{enumerate}
\end{enumerate}








\end{document}