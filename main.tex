%&pdflatex

\documentclass[12pt,letter]{article}
%%%%%%%%%%%%%%%%%%%%%%%%%%%%%%%%%%%%%%%%%%%%%%%%%%%%%%%%%%%%%%%%%%%%%%%%%%%%%%%%
\usepackage{booktabs, caption, subcaption, float}
\usepackage[latin1]{inputenc}
\usepackage{longtable}
\usepackage{graphics}
\usepackage{geometry}
\usepackage{graphicx}
\usepackage{amsfonts, amssymb, amsmath}
\usepackage{natbib, multibib}
\usepackage{theorem}
\usepackage{setspace}
\usepackage[normalem]{ulem}
\usepackage{caption}
\usepackage{tabularx}
\usepackage{epstopdf}
\usepackage{rotating}
\usepackage{placeins}
\usepackage[makeroom]{cancel}
\usepackage{varioref}
\usepackage{hyperref}

\setcounter{MaxMatrixCols}{10}
% TCIDATA{OutputFilter=Latex.dll} TCIDATA{Version=5.50.0.2960}
% TCIDATA{<META NAME="SaveForMode" CONTENT="1">}
% TCIDATA{BibliographyScheme=BibTeX} TCIDATA{LastRevised=Monday, May
% 11, 2015 19:08:26} TCIDATA{<META NAME="GraphicsSave" CONTENT="32">}

\theoremstyle{break} \theorembodyfont{\normalfont\itshape}
\newtheorem{thm}{Theorem}
\theoremstyle{break}
\theorembodyfont{\normalfont\itshape}
\newtheorem{corrollary}{Corollary} \theoremstyle{break}
\theorembodyfont{\normalfont\itshape} \newtheorem{prop}{Proposition}
\theoremstyle{break} \theorembodyfont{\normalfont\itshape}
\newtheorem{lemma}{Lemma} \theoremstyle{break}
\theorembodyfont{\normalfont\itshape} \newtheorem{algm}{Algorithm}
\theoremstyle{break} \theorembodyfont{\normalfont\itshape}
\newtheorem{definition}{Definition} \theoremstyle{break}
\theorembodyfont{\normalfont\itshape}
\newtheorem{condition}{Condition} \theoremstyle{break}
\theorembodyfont{\normalfont\itshape}

\renewcommand{\baselinestretch}{1.5}
\setlength{\hoffset}{-0.4in}
\setlength{\textwidth}{6.8in}
\setlength{\textheight}{9.3in}
\setlength{\topmargin}{-0.5in}



\begin{document}

\author{Tarek A. Hassan\thanks{\textbf{Boston University}, NBER, and
    CEPR; Postal Address: 270 Bay State Road, Boston, MA 02215, USA;
    E-mail: thassan@bu.edu.} \and  Tony
  Zhang\thanks{\textbf{%
      Boston University, Questrom School of Business}; Postal Address:
    595 Commonwealth Avenue, Boston, MA 02215, USA; E-mail:
    tzhang0@bu.edu.} } \title{The Economics of Currency Risk Premia\thanks{}}
\date{ \bigskip July 2020}
\maketitle

% +Title

\thispagestyle{empty}


\vspace{-0.7cm}


\setstretch{1.0}



\begin{abstract}
This article reviews the literature on currency risk with a focus on its macroeconomic origins and implications. A growing body of evidence shows that countries with safer currencies enjoy persistently lower interest rates, a lower required return to capital, and accumulate relatively more capital than countries international investors perceive as risky. While earlier research has focused mainly on the role of currency risk in generating violations of uncovered interest parity and other financial anomalies, more recent evidence also points to important implications for the allocation of capital across countries, the efficacy of exchange rate stabilization policies, the sustainability of trade deficits, and the spill-over of shocks across borders.
\end{abstract}


\bigskip

\bigskip {\noindent \textbf{JEL classification:} }

{\noindent \textbf{Keywords:}  }

\pagebreak

\setstretch{1.4} \setcounter{page}{1}


\section{Introduction}


A key tenet in economics is that the degree to which firms should be willing to invest in a given project depends crucially on the required rate of return to capital: a price-taking firm should install just enough capital so that the marginal product of capital, $MPK_i$ equals the required rate of return to capital,$r_i$, \begin{equation}MPK_i=\underbrace{r^f+RP_i}_{ r_i}.\label{eq_one}\end{equation} 
This equation is the point of departure for many classic questions in economics. Students of asset pricing are taught that a firm's $r_i$ has two components; a risk-free part, $r^f$, and a risk-premium, $RP_i$, that depends on the firm's risk characteristics. One of the classic puzzles in asset pricing is why $RP_i$ is so large relative to $r^f$ (the equity premium puzzle). Monetary economists are interested in the Federal Reserve's power to manipulate $r^f$, while students of economic growth often take differences in the $MPK_i$ as a measure of inefficiencies in the allocation of capital across countries.

In this article we argue that recent insights from the study of currency risk premia have important lessons for how we should think about equation (\ref{eq_one}), and by extension, for its key applications in asset pricing, macroeconomics, and economic growth. The simplest, and perhaps and most important, of these lessons is that countries differ dramatically in their risk-free interest rates. These differences in risk-free interest rates are large (on the same order of magnitude as the equity premium puzzle), appear to be highly persistent over time (lasting for decades rather than years), and cannot be explained by predictable depreciations, government default, or differences in inflation rates. Instead, these differences in interest rates appear intimately linked to the risk characteristics of the country's exchange rate, and to its exchange rate regime.   

Figure XX shows this fact using the New Zealand Dollar and the Japanese Yen as an example. It plots the difference in the two currencies' risk-free interest rates over time (we will discuss below how one can measure such risk-free rates and compare them across countries). The Figure shows that between 19XX and 20XX, the New Zealand always had a higher risk-free rate than Japan, on average this difference was about 6pp. When we adjust for movements in exchange rates over the period, the difference in returns on the two countries currencies is XXpp, and XXpp when we adjust for differential inflation. Over the same time period, the equity premium on US stocks was about XXpp.

think its risk

appears to have real effect. For example K/Y in New Zealand is XX percent lower than in Japan.


\section{Risk-free Interest Rates, Exchange Rates, and Currency returns}

Measuring Risk-free interest rates, show data

Hassan and Mano facts, Violations of UIP, Fama 84

\section{Interest Rate Differentials are Differences in Risk Premia}
\subsection{Theory}
\begin{enumerate}
    \item Bare-bones model, no differences in r when risk-neutral
    \item Get differences in r when risk averse
    \item Pretty much any asymmetry you put into a standad macro model will produce differences in r
\end{enumerate}

\subsection{Evidence that currency returns depend on risk}
\begin{enumerate}
\item[-] Some currencies are safer than others
\item SIMPLE PICTURE
\item Campbell, Meideiros and Viceira (2010)
\item[-] Risk factors: People in finance have shown this formally
\item Lustig and Verdelhan (2007)
\item Brunnermeier, Nagel and Pedersen (2008)
\item Lustig, Roussanov and Verdelhan (2011)
\item Burnside, Eichenbaum, Kleshchelski and Rebelo (2011)
\item Menkhoff, Sarno, Schmeling, and Schrimpf (2012)
\item Lustig, Roussanov and Verdelhan (2014)
\end{enumerate}

\subsection{Microfoundations}
These differences in risk premia arise naturally in macro models.
\begin{enumerate}
\item Hassan (2013)
\item Gabaix and Maggiori (2015)
\item Ready Roussanov Ward (2017)
\item Richmond (2019)
\item Farhi and Gabaix (2016)
\item Della Corte, Riddiough and Sarno (2016)
\item Tran (2013)
\item Powers (2015)
\item Wiriadinata (2020)
\end{enumerate}



If there is indeed something to this story, it changes how we think
about central issues in macro.

\section{Currency Risk and the allocation of capital across countries}
There are large differences in capital accumulation across countries:
\begin{enumerate}
\item Lucas (1988)
\item Caselli and Feyrer (2007)
\item Monge-Naranjo, Sanchez and Santaeualia-Llopis (2018)
\end{enumerate}
These differences may be explained by risk premia:
\begin{enumerate}
\item di Giovanni, Kalemli-Ozcan, Ulu, Baskaya (2019)
\item Richers (2020)
\item Hassan, Mertens and Zhang (2016)
\item David, Henriksen and Simanovska
\end{enumerate}
[picture]
\section{Currency Risk and Exchange Rate Regimes}
Q: Are there other papers that argue policies can manipulate risk premia?

\section{Dynamics}
\subsection{Quantitative literature}
\begin{enumerate}
\item Colacito and Croce (2011)
\item Gourio, Siemer and Verdelhan (2013)
\item Colacito, Croce, Ho and Howard (2018)
\end{enumerate}
\subsection{Theories written for FPP}
Heyerdahl-Larsen
Statopoulous
Verdelhan
Backus Foresi Telmer
\subsection{Measurement of risk}
Maybe measurement of risk
- implied vol
- commercial indices
- newspapers
- conference call transcripts
\subsection{Global financial cycle}

\section{Capital flows and the allocation puzzle}
Capital flows run counter to neo-classical model:
\begin{enumerate}
\item Gourinchas and Jeanne (2013)
\item[-] Current explanations:
\item Differences in financial development (Ju and Wei, 2006;
  Caballero, Farhi and Gouinchas, 2008; Mendoza, Quadrini and
  Rios-Rull, 2009)
\item Differences in factor utilization (Jin, 2012)
\item Differences in contracting frictions (Aguiar and Amador, 2010)
\end{enumerate}

\section{Dynamics of Currency Risk Premia}


\section{Unsorted}
\begin{enumerate}
\item Gourinchas and Rey (2007)
\item Govillot, Rey and Gourinchas (2010)
\item[-] Implications for equilibrium portfolios and global imbalances
\item Miranda-Agrippino and Rey (2020)
\item[-] Global financial cycles
\item Boccola and Lorenzoni (forthcoming)
\item[-] Debt crises
\item Du and Schreger
\item[-] Corporate balance sheets
\item[-] Special role for the dollar
\item Hassan, Mertens and Zhang (2020)
\end{enumerate}



\end{document}