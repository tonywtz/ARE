
\documentclass{ar-1col}

\usepackage{amsmath}
\usepackage[comma]{natbib}
\usepackage{url}
\setcounter{secnumdepth}{4}

% Metadata Information
\jname{Xxxx. Xxx. Xxx. Xxx.}
\jvol{AA}
\jyear{2020}
\doi{10.1146/((please add article doi))}



\begin{document}

% Page header
\markboth{Hassan and Zhang}{The Economics of Currency Risk}

% Title
\title{Title: Subtitle}


% Authors, affiliations address.
\author{Tarek A. Hassan,$^1$ and Tony Zhang$^2$ \affil{$^1$Department
    of Economics, Boston University, Boston, USA, 02213; email:
    thassan@bu.edu} \affil{$^2$Board of Governors of the Federal
    Reserve System, Washington, USA, 20551; email:
    tony.zhang@frb.gov}}



\title{The Economics of Currency Risk\thanks{}}



\begin{abstract}
  This article reviews the literature on currency risk with a focus on
  its macroeconomic origins and implications. A growing body of
  evidence shows that countries with safer currencies enjoy
  persistently lower interest rates, a lower required return to
  capital, and accumulate relatively more capital than countries
  international investors perceive as risky. While earlier research
  has focused mainly on the role of currency risk in generating
  violations of uncovered interest parity and other financial
  anomalies, more recent evidence also points to important
  implications for the allocation of capital across countries, the
  efficacy of exchange rate stabilization policies, the sustainability
  of trade deficits, and the spill-over of shocks across borders.
\end{abstract}


% Keywords, etc.
\begin{keywords}
  keywords, separated by comma, no full stop, lowercase
\end{keywords}
\maketitle

% Table of Contents
% \tableofcontents

\section{Introduction}


A key tenet in economics is that the degree to which firms should be
willing to invest in a given project depends crucially on the required
rate of return to capital: A price-taking firm should install just
enough capital so that its marginal product of capital, $MPK_i$,
equals the required rate of return to capital, $r_i$,
\begin{equation}
  MPK_i=\underbrace{r^f+RP_i}_{r_i}.
  \label{eq_one}
\end{equation} 
This equation is the point of departure for many classic questions in
economics. Students of asset pricing are taught that a firm's $r_i$
has two components; a risk-free part, $r^f$, and a risk-premium,
$RP_i$, that depends on the firm's risk characteristics. One of the
classic puzzles in asset pricing asks why $RP_i$ is so large relative to
$r^f$ (the equity premium puzzle). Monetary economists are interested
in the Federal Reserve's power to increase or decrease investment by
manipulating $r^f$, while students of economic growth often take
differences in $MPK_i$ as a measure of inefficiencies in the
allocation of capital across countries, sectors, and firms.

In this article we argue that recent insights from the study of
currency risk premia have important lessons for how we should think
about Equation \eqref{eq_one}, and by extension, for its key
applications in asset pricing, macroeconomics, and economic growth.
The simplest, and perhaps and most important, of these lessons is that
countries differ dramatically in their risk-free interest rates. These
differences in risk-free interest rates are large (on the same order
of magnitude as the equity premium puzzle), appear to be highly
persistent over time (lasting for decades rather than years), and
cannot be explained by predictable depreciations, government default,
or differences in inflation rates. Instead, these differences in
interest rates appear intimately linked to the risk characteristics of
the country's exchange rate, and to its exchange rate regime.

\begin{figure}
  \centering
  \caption{Risk-free Interest Rates}
  \includegraphics[width=0.7\textwidth]{Exhibits/Figure_FP12M_JPYNZD.pdf}
  \label{fig:fp}
\end{figure}
Figure \ref{fig:fp} shows the difference in the risk-free interest
rates of the New Zealand Dollar and the Japanese Yen as an example (we
will discuss below how one can measure such risk-free rates and
compare them across countries). The figure shows that the New Zealand
dollar had a higher risk-free rate than the Japanese Yen in
\textit{every} month between January 1997 and December 2018. On average,
this difference was about 4.90pp on an annualized basis. When we
adjust for movements in exchange rates over the period, the difference
in returns on the two countries currencies is 4.39pp -- meaning that a
US investors who borrowed in Yen and lent in New Zealand Dollars on
average made a return of XX percent over this period. (For comparison,
the equity premium on US stocks was about XXpp during the same
period.)

We now know that such highly persistent differences in interest rates,
similar to those between New Zealand and Japan, are common in the
data, even among developed economies. These large differences in
interest rates do not appear to be equalizing over time and translate
into large differences in returns investors can earn when investing in
these currencies.

Why would $r^f$ differ permanently across countries? The emerging
consensus in the literature is that the most likely explanation are
currency risk premia -- the idea that some currencies are safer
investments than others.

A useful way of thinking about this problem is to take the perspective
of a retail investor in a third country, say in Hong Kong. As is
common in many countries, Hong Kong-based banks regularly offer
savings accounts denominated in multiple currencies, so that our
fictitious investor might have the option to invest in yen at a
deposit rate of 0.1\% or in New Zealand Dollars at a rate of 3.0\%.
How might she decide between these two options? Since both accounts
are with the same bank, any likelihood of sovereign default is
irrelevant. Similarly, our Hong-Kong based investor does not care
about inflation in these two faraway countries. Instead, the only
relevant factors for her choice between these two investment is the
difference in interest rates and the stochastic behavior of the yen-to
New Zealand dollar exchange rate.

Because changes in exchange rates are largely unpredictable over short
horizons, our investor should not expect either currency to depreciate
over the coming year, leaving only one relevant consideration:
covariance.\footnote{A regression of the Japanese Yen to New Zealand Dollar
exchange rate on their interest rate differential yields an R-squared of
0.XX} Which of the two currencies does our investor trust to
retain value in a possible recession or crisis? Intuitively, she might
think that Japanese yen are a safer bet -- and she would be right.

Along with a number of other so called ``safe-haven currencies,'' the
Japanese yen tends to appreciate relative to the New Zealand dollar
during large international recessions and crises. 
\begin{figure}[htp!]
  \centering
  \caption{New Zealand Dollar - Japanese Yen Exchange Rate}
  \includegraphics[width=0.7\textwidth]{Exhibits/Figure_FX_JPYNZD.pdf}
  \label{fig:spot}
\end{figure}
Figure
\ref{fig:spot} shows some evidendence of this pattern by plotting the New Zealand dollar - Japanese yen exchange
rate in terms of dollars per yen (so that an increase in the exchange rate
indicates yen appreciation). The shaded areas highlight three distinct
periods of global economic turmoil: The early 1990s recession (1990 -
1993), the Asian financial crisis (1997 - 1998), and the Great
Recession (2007 - 2009). In each of these periods, the yen appreciated
markedly against the New Zealand Dollar. If these appreciations during
periods of economic turmoil are part of a broader pattern, investors
should naturally consider the Japanese yen the safer currency -- and
if yen are a safer
store of value, it might make sense to accept a lower deposit rate.
That is, international investors might be willing to lend at lower
rates in currencies they expect to retain value when times are bad.


As it turns out, this simple intuition has a lot of support in the
data from international bond and derivatives markets. For example, a
seminal paper by \citet{LustigVerdelhan2007} shows that currencies
with low interest rates tend to appreciate when US consumption
growth is low, and depreciate when US consumption growth is high. That
is, there is direct evidence that currencies with low interest rates
appreciate when times are bad for US consumers, making these
currencies a safer store of value for investors.

In addition to this empirical evidence, the theoretical work on
currency risk has identified a number of theoretical reasons to expect
the emergence of safe-haven currencies and long-lasting differences in
interest rates. In a nutshell, persistent differences in countries' currency risk premia arise naturally in
a wide range of international macro models. For example,
\citet{Hassan2013} shows that simply allowing for some economies to be
larger than others within a standard international real business cycle
model is sufficient to generate long-term differences in interest
rates between countries, because the currencies of larger countries
tend to appreciate when world-wide output is low. That is, even within
the most canonical, frictionless, international macro models currency
risk premia tend to arise naturally in equilibrium. Other authors have
similarly pointed to the emergence of currency premia in models with
intermediary capital constraints, trade costs, and differences in
resource endowments, among others.

We survey the rapidly growing empirical and theoretical literature on
currency risk premia in detail in sections XX and XX. Because the
initial focus of this literature was mainly on resolving asset pricing
anomalies, many of its key papers tend to use technical finance
language. For this reason we attempt to keep this review as
non-technical as possible, focusing as much as possible on the
underlying economics.

A major difficulty this literature shares with a broader literature in
asset pricing is that models with conventional preferences tend to
produce risk premia that are quantitatively small. That is, although a
number of papers have identified compelling reasons why, for example,
the interest rate in Japan should be lower than that in New Zealand,
most of these models suggest that it should be lower by something on
the order of 0.04pp rather than the 4.23pp we measure in the data. In
this sense, the literature is running into a quantitative ``interest
differential puzzle,'' which is in some ways analogous to the equity
premium puzzle. Both puzzles fundamentally struggle with the
prediction of models with standard preferences that risk premia should
be small given the relatively small aggregate variation in consumption
growth we measure in the data. Quantitative research on currency risk
is thus a major area for future research.

Although the literature on currency risk premia has proliferated in
recent years, it has perhaps been less successful at making clear the
relevance of its findings beyond the financial context, a gap we hope
to partially fill with this article.

Perhaps the most immediate implication of currency premia for the real
economy is for capital accumulation: If countries differ persistently
in $r^f$, then those with higher interest rates have a persistently
higher cost of capital and thus, according to equation (\ref{eq_one}) 
should produce with relatively less capital. Returning to our example from
Figure \ref{fig:fp}, it turns out that indeed the capital-to-output
ratio in New Zealand is 22 percent lower than in Japan, suggesting
that, indeed, the marginal product of capital is larger in New Zealand
than it is in Japan. More generally, countries with persistently
higher interest rates appear to have higher marginal products of
capital in the long-run. This simple insight has direct implications
for several strands of the macroeconomic literature.

The first is for the so-called Lucas-puzzle \citep{Lucas1990}, which
posits that, over long periods of time, capital-to-output ratios do
not appear to be equalizing across countries, and in particular, not
enough capital appears to be flowing to developing nations to equalize
the marginal product of capital. If indeed some currencies are
permanently riskier than others, then currency risk is one possible
factor preventing such equalization. Policies that reduce the perceived riskiness of a given currency could then also contribute to equalizing the marginal product of capital across countries.

A second, related, implication is for a large literature that focuses
on assessing the efficiency of the allocation of capital across
countries \citep{HallJones1997, CaselliFeyrer2007}. A basic assumption
in this literature is that $r^f$ is equalized across countries, so
that systematic deviations from this required rates can (partially) be
attributed to inefficiencies. However, if there are fundamental
(efficient) reasons for $r^f$ to differ due to differing currency (and
country) risk characteristics, some of these calculations will have to
be revisited.

Third, an active literature in international finance has studied the
propensity of firms and countries to borrow in foreign currency
\citep{DuSchreger2016, KalemliOzcanetal2019}. If lending in a
low-interest-rate currency is safer than lending in a
high-interest-rate currency, then the opposite is true for borrowing.
That is, firms that borrow in dollar, yen or another safe-haven
currency may be loading up on systematic risk -- a price they pay for
enjoying lower rates. This systematic risk may then be a concern threatening the resiliency of firms balance sheets and governments' solvency during major crises.

Aside from issues surrounding capital accumulation and investment,
long-lasting differences in interest rates across countries may also
change how we think about two major XXX. The United States and a
number of other countries is running a persistent current account
deficit, the sustainability of which depends crucially on its ability
to borrow cheaply in international markets \citep{GourinchasRey2007}.
If there are fundamental economic reasons why the US dollar is a safer
currency than many others, then lower US interest rates are here to
stay, potentially enabling the United States to sustain trade and
current account deficits in perpetuity.

If there is indeed something to this story, it changes how we think
about central issues in macro.

4) Exchange rate regimes and risk 5) Dynamics, capital flows and risk

FIXMETH CONTINUE ONCE WE KNOW HOW WE END THE PAPER

\section{Risk-free Interest Rates, the Carry Trade, and Failures of
  Uncovered Interest Parity}

Early work on currency risk began as an attempt to understand three
puzzling empirical facts in currency markets. The first, often referred to as the \textit{forward premium puzzle}, is that
regressions of changes in the exchange rates on differences in
interest rates yield a coefficient smaller than one, suggesting that,
on average, high-interest-rate currencies do not depreciate enough to
wipe out any differences in interest rates \citep{Bilson1981, Fama1984}.
The second is that investors in the \textit{carry trade} appear to be
making money by borrowing in currencies with low interest
rates and lending in currencies with high interest rates. The third puzzling fact is that differences in 
interest rates and currency returns are highly persistent, so that the same
countries tend to have high or low interest rates on a very long-term
basis.



\begin{textbox}[h]\section{Measuring Risk-free Interest Rates and Currency Returns}

Before we characterize these three facts in more detail, it is worth
taking a moment to think about how to measure and compare returns and risk-free
interest rates across currencies. In the United States we often regard
T-Bills as risk-free. However, even if we assume the US government
cannot default, yields on government debt of many other countries are
almost certainly contaminated with the possibility of government
default.

For this reason, most of the recent literature follows \cite{LustigRoussanovVerdelhan2011} by
constructing synthetic risk-free rates using currency forward
contracts and the Covered Interest Parity (CIP).\footnote{A forward
  contract is a rate at which a bank agrees to exchange one currency
  for another at a pre-specified date in the future.} To understand
CIP, consider again our example of Japan and New Zealand, and suppose
there is a risk-free savings account in each country. A Japanese
investor with access to both these accounts and to currency forward
contracts then has two ways of making a risk-free investment in yen.
First, she can simply save at the yen deposit rate. Alternatively, she
can convert yen into New Zealand dollars at the spot exchange rate,
invest the New Zealand dollars at the local savings rate, and sign a
forward contract to exchange her New Zealand dollars back to yen at the
end of her investment period. Both strategies are risk-free since the
spot and forward exchange rates are known today. Therefore, if there
is no arbitrage, the two strategies must yield the same return:
\begin{equation}
  \underbrace{1 + r_{JPY}}_{\text{yen risk-free rate}}
  = \underbrace{
    (1 + r_{NZD}) \times \frac{S}{F}
  }_{\text{Implied yen risk-free rate}}.
\end{equation}
Taking logs on both sides of this equation and rearranging shows
$r_{JPY}-r_{NZD}=s-f$ --- The difference in risk-free interest
rates must equal the difference between the (log) spot and forward
exchange rates. This difference is known as the forward premium. As a
result, we can reliably measure risk-free interest rate differentials
from forward prices, even if a given government's ability to repay its
debts may be in question.\footnote{For this reason, commercial providers of the
carry trade also tend to implement it by buying and selling forward
contracts, rather than by trading corporate or government bonds.}.\end{textbox}


All three of these facts have in common that they describe different violations of uncovered interest parity (UIP), that is, for a given pair of currencies $h$ and $f$, the expected rate of depreciation is usually not equal to the interest differential, $$E\Delta s \neq r^f_h-r^f_f .$$ An extensive literature has documents these violations of UIP and a number of related trading strategies. Hodrick (1987), Froot and Thaler (1990), Engel (1996), Lewis (2011), and Engel (2014) provide excellent surveys. 

One difficulty for students of this literature is that it evolved simultaneously as an attempt to understand exchange rate dynamics (e.g. the forward premium puzzle) and asset pricing anomalies (e.g. the carry trade), and thus often alternates between regression-based and portfolio-methods that are not necessarily directly comparable to each other. We next give a brief overview using the unified approach by \citet{HassanMano2015}. This allows us to show the main empirical facts in a form where regressions map one-for-one into portfolios, while also abstracting from many of the details uncovered in the two branches of the literature.

Thus, one way to implement a carry trade is to calculate risk-free 
rates for each country $i$ at the start of each month $t$ during an 
investment period $t = 1, ..., T$, and then form a portfolio where 
each currency is weighted by the difference between its risk-free rate 
and the average risk-free rate of all currencies. The return on this 
portfolio is:
\begin{equation}
  \label{eq_carry}
  \textstyle\sum_{i}\sum_t\left[ rx_{i,t+1}\left( r^f_{it}-\overline{r}^f_{t}\right) \right] ,
\end{equation}%
where, $rx_{i,t+1}=r^f_i-r^f_{USD}+\Delta s$ is the return to 
borrowing in the home currency and lending in foreign currency $i$ at
between $t$ and $t+1$.

\citet{HassanMano2015} implement this strategy for 39 currencies between 
1986 and 2010 and obtain an annualized mean return of 5.45 percent with 
a Sharpe Ratio of 0.69. For comparison, the Sharpe ratio of the U.S. 
equity market during the same time period is XX. In this sense, the carry 
trade is a profitable strategy that demands an explanation: are carry 
traders being compensated for taking on risk? If so, what kind of 
risk are they taking?

We get the third fact mentioned above by trying to understand whether we
obtaining the carry trade returns as a result of re-sorting currencies in 
our portfolio by their risk-free rates in each month. 
\citet{HassanMano2015} show that the incremental gain to re-sorting the 
carry trade portfolio every month is usually not statistically 
distinguishable from zero. Instead, the authors find that their carry 
trade strategy would have obtained 70\% of the same returns if they had 
sorted their portfolio just once at the beginning of their investment 
period using an initial three years of data, and then never changed their
portfolio afterwards.

The carry trade return in (\ref{eq_carry}) can be interpreted as the 
covariance between currency returns at $t+1$ and interest differentials 
at time $t$. \citet{HassanMano2015} therefore estimated the returns to 
the carry trade by running a predictive regression of the form:
\begin{equation}
  rx_{i,t+1}-\overline{rx}_{t+1}=\beta^{ct}\left(
    r^f_{it}-\overline{r}^f_{t}\right) +\epsilon
  _{i,t+1}^{ct}. 
  \label{eq_ct}
\end{equation} 
In the sample of 39 currencies, $\hat{\beta}^{ct}=0.67 (s.e.=0.16)$, 
implying currencies with high interest rates --- relative to other 
currencies in the same month --- on average pay significantly higher 
returns.


% where $\beta ^{ct}$ can be interpreted as the elasticity of currency
% risk premia with respect to interest rate differentials conditional on
% a time fixed effect. 


The results from estimating equation \eqref{eq_ct} also imply that for 
every dollar carry traders make on interest differentials, 1-0.67=0.33 
dollars are wiped out by predictable depreciations. In other words,
high-interest-rate currencies depreciate on average, but not by 
enough to eliminate the interest differential. We should note that 
there has been some confusion in the literature on this final point. 
In many older papers, authors show regressions of exchange rates on 
interest rate differentials with a country-specific intercept:
\begin{equation}
    \Delta s_{i,t+1} 
    = \alpha_i + \gamma \left(r^f_{i, t} - r^f_{US, t}\right) + \nu_{i, t+1},
\label{eq_fama} 
\end{equation}
and interpret the fact that the slope coefficient $\gamma$ is often 
negative as evidence that investors expect high-interest-rate 
currencies to appreciate instead of depreciate. This finding has 
prompted a considerable theoretical literature trying to rationalize 
extremely volatile currency risk premia that are negatively correlated 
with predicted depreciations (the ``Fama conditions''). However, this
interpretation is incorrect, because a regression with currency fixed
effects, such as equation \eqref{eq_fama} is not predictive -- investors at
time $t$ do not know what each currency's fixed effect is. Once one
corrects for this discrepancy between realized and predicted
appreciations, one recovers results consistent withe the view that, on
average, investors expect high-interest-rate currencies to depreciate,
not appreciate.

To summarize, the carry trade is highly profitable because there are 
highly persistent differences in risk-free interest rates across 
currencies that are only partially reversed by predictable 
depreciations. Currencies with higher interest rates on average pay 
higher returns to international investors than currencies with lower 
interest rates. In addition, there is some evidence that expected 
currency returns move over time with variation in interest rates, 
so that currencies with unusually high interest rates also pay 
unusually high returns.\footnote{In addition to the three puzzling
facts discussed in this section, there are a number of empirical 
puzzles that we do not have the space to discuss in detail. 
\citet{LRV2014}, \citet{ChinnMeredith2004}, \citet{LustigStathopoulosVerdelhan2019}
\citet{}

Engel AER, Valchev JMP, 
Lustig and richmond some currencies appreciate when they have 
high interest rates}

Taken together, [questions that organize the next sections]



\section{Interest Rate Differentials as Differences in Risk Premia}



\subsection{Reduced-form Evidence}

To formally attribute the returns of currency traders to risk, we need
to borrow some technology from the broader asset pricing literature.
A large empirical asset pricing literature constructs
\emph{risk-factors} to attribute cross-sectional differences in asset
returns to different sources of risk by looking for comovement in
asset returns \citep{Fama1976}. Typically, researchers sort assets
based on some characteristic of interest, and then divides asset into
a small number of portfolios based on the sort. The first portfolio
typically contains assets with the lowest values of the characteristic
of interest, and the last portfolio typically contains assets with the
highest values. A risk factor is then constructed by taking the
difference in returns between the fist and last portfolio. Researchers
determine a risk factor explains asset returns by running regressions
of returns on the risk factor, and showing that assets with higher
regression coefficients on the risk factor also obtain higher expected
returns.\footnote{For example, \citet{FamaFrench1992} constructed two
  risk factors by sorting U.S. equities into portfolios based on
  market capitalization and book-to-market ratio, and showed
  portfolios of stocks with greater exposure to these two risk factors
  obtained higher average returns.}

% A key innovation of the empirical asset pricing literature is also
% to analyze portfolio returns rather returns than on individual
% assets. Again, assets are sorted into portfolios based on some
% dimension of interest. Intuitively, averaging the returns of assets
% within portfolios eliminates diversifiable and asset-specific risks.
% The remaining variation in returns across portfolios should better
% capture the risk-return trade-off specifically from differing along
% the dimension of interest.

In a seminal paper, \citet{LustigRoussanovVerdelhan2011} applied these
asset pricing techniques to exchange rates and provided systematic
evidence that the greater returns obtained from investing in
high-interest-rate currencies result from greater risk exposure. For
every month between November 1983 and December 2009, the authors
sorted currencies into 6 portfolios based on their risk-free rate
differential relative to the U.S. dollar. The first portfolio
contained currencies with the lowest risk-free rates and the last
portfolio contained currencies with the highest risk-free rates. The
authors constructed a \emph{carry trade} risk factor by taking the
difference in the returns between the portfolio containing the
highest-interest-rate currencies and the portfolio containing the
lowest-interest-rate currencies.

\citet{LustigRoussanovVerdelhan2011} showed that differential exposure
to their carry trade risk factor accounted the expected returns from
investing in different currencies. The authors regressed the returns
of their 6 currency portfolios on their carry trade risk factor. The
portfolios with higher-interest-rate currencies obtained a higher
average return and also covaried more positively with the risk factor.
An increase in the regression coefficient on the carry trade risk
factor from 0 to 1 was associated with a large and highly significant
increase of 5.5 percent per year.

In this sense, \citet{LustigRoussanovVerdelhan2011} identified a
common source of risk in currency markets, and showed that exposure to
this common source of risk explained cross-sectional differences in
currency returns. However, the major drawback of this asset pricing
method is that it does not reveal the ultimate source of risk. In
other words, we know exchange rates move with each other, but we do
not know the economic forces that drive this comovement.

Hence, many researchers have tried to identify relationships between
currency returns and macro-financial variables to understand the
source of currency risk. \citet{LustigVerdelhan2007} showed that
low-interest-rate currencies provide a hedge against U.S. consumption
growth risk. Low-interest-rate currencies systematically appreciate
when U.S. consumption growth is low, and high-interest-rate currencies
tend to depreciate when U.S. consumption growth is low. Thus, the U.S.
investor can hedge against periods of low consumption growth by
investing in a portfolio of low-interest-rate currencies. U.S.
investors find this hedging property useful, and therefore accept a
lower rate of return.

Moreover, low-interest-rate currencies tend to appreciate during
periods of financial turmoil, whereas high-interest-rate currencies
tend to depreciate. \citet{LustigRoussanovVerdelhan2011} and
\citet{CampbellMedeirosViceira2010} showed low-interest-rate
currencies tend to appreciate whenever equity markets are volatile.
Consistent with this evidence, \citet{Menkhoffetal2012} measure of
periods financial market turmoil using exchange rate data, and show
low-interest-rate currencies provide a hedge against periods of high
exchange rate volatility.

Literature on crash risk:
\begin{itemize}
\item \citet{Brunnermeieretal2009} studied eight major currencies.
  Higher interest rate currencies exhibited greater chance of large
  devaluations (i.e. crash risk).
\item Jurek2014 - Constructs carry trades without crash risk to
  quantify the contribution of crash risk to carry trades. At most
  1/3.
\item Farhietal2015 - Similar to Jurek. How much of the carry trade is
  driven by crash risk? Also arrives at around 1/3.
\item Lewis2011 - peso problems (no risk premia, simply mismeasuring
  average returns due to small samples)
\end{itemize}

Each of these papers highlight different ways in which
high-interest-rate currencies may be riskier than low-interest-rate
currencies. However, the common thread among these papers is that the
persistent differences in returns to investing in various currencies
reflect persistent differences in the stochastic properties of their
exchange rates. Under this interpretation, currencies yielding lower
returns are safer currencies that tend to appreciate during periods of
economic distress.


\subsection{Theory: Microfoundations of Safe Haven Currencies}

Prompted by this empirical evidence, the theoretical literature has
identified several fundamental economic forces that may make a given
currency safer or riskier from the perspective of global investors.
Although there is an ongoing debate on which of these forces may be
most important in practice, many of of these microfounded models share
a common structure, which we can summarize using just a few key
equations.

Consider a world economy in which international assets are priced by
the marginal utility of an international investor, $\lambda_T$, which
will be our measure of ``good'' and ``bad'' times. Times are good when
$\lambda_T$ is low, and times are bad when it is high. In different
models, $\lambda_T$ may be the marginal utility of consumption of
traded goods (the part of consumption that his shared
internationally), the marginal utility of consumption of a key
investor who intermediates between segmented international markets, or
the capital constraint of such an intermediary. Households consume a
country-specific final good, the price of which (accounted for in a
common unit) depends on $\lambda_T$ and a country-specific shock,
$x^n$,
\begin{equation}
  p^{n} = a\lambda_{T} + b x^{n}.  
  \label{eq_RF}
\end{equation}%
The country-specific shock interchangeably may stem from a
country-specific supply, demand, or monetary shock; in other words, it
is a stand-in for any factor that affects the price of consumption in
one country more than in others. The higher $x^{n}$, the higher the
price of domestic consumption relative to that in other countries. For
simplicity, assume $\lambda _{T}\sim N(0,\sigma^2_{\lambda_{T}})$ and
$x^{n} \sim N(0,\sigma^2_x) $ are normally distributed, not
necessarily independent, shocks and $a$ and $b$ are positive constants.

The real exchange rate, $S^{f, h}$ between two countries $f$ and $h$ 
is the relative price of their respective final goods. In logs,
\begin{equation}
  s^{f,h} 
  = p^f - p^h 
  = b(x^f - x^h),
\label{eq_RER}
\end{equation}
where the second equality substitutes Equation \eqref{eq_RF}. Because 
of $x^{n}$, the relative price of consumption may differ between the 
two countries, allowing the real exchange rate to move. When the price of
consumption in country $f$ increases, its consumption bundle
appreciates relative to country $h$. In this sense, we can speak of
``currencies'' even without formally introducing money into the model.

By definition, the risk-free bond in country $h$ pays $p^h$ with
certainty, while that in country $f$ pays $p^f$ with certainty, so
that each country's risk-free bond pays exactly one unit of that
country's consumption bundle. Importantly, because the real exchange
rate may move around, country $h$'s risk-free bond is not risk-free
from the perspective of households in country $f$, and vice versa. For
this reason, one country's risk-free bond may be more expensive than
the other country's risk-free bond. The risk-free rate of interest is
the yield to maturity of the risk-free bond, so that the country with
the more expensive risk-free bond automatically must have the lower
risk-free rate of interest. In this sense, any model with a variable
real exchange rate as in equation (\ref{eq_RER}) may produce
differences in risk-free interest rates across countries.

Different theories of such differences in interest rates apply
elementary asset pricing to equation (\ref{eq_RER}). One can show that
the log expected return to borrowing in country $h$ and to lending in
country $f$ is
\begin{equation}
  r^{f} + \Delta \mathbb{E} s^{f,h} - r^{h} =cov\left( \lambda _{T},p^{h}-p^{f}\right),
  \label{eq_UIP_RF}
\end{equation}%
where $r^{n}$ is the risk-free interest rate in country
$n$.\footnote{$\Delta\mathbb{E}s^{f,h}$ is defined as the logarithm of
  the ratio of the countries' expected real price changes. See
  Appendix \ref{Appendix_ReducedFormResults} for a formal derivation.}
This statement means a currency that tends to appreciate when
$\lambda_T$ is high pays a lower expected return and, if
$\Delta \mathbb{E} s^{f,h}\approx0$ (as is the case in the data),
therefore must has a lower risk-free interest rate. That is, a
currency that appreciates in bad times (for example, in times when
consumption goods are expensive everywhere) provides a hedge against
worldwide consumption risk and pays lower returns in equilibrium.

Equations \eqref{eq_RF} and \eqref{eq_UIP_RF} are the main ingredients
of risk-based models of unconditional differences in interest rates
across countries, where different approaches have identified different
reasons why the countries differ in the degree to which their price
indices covary with $\lambda_T$.

\paragraph*{Country Size}




FIXMETH Describe
\begin{enumerate}
\item \citet{Hassan2013}

  \begin{equation} \lambda_{T}^\ast = -c \sum_{n} \theta^n x^n,
    \label{eqn:lambdat2NP}
  \end{equation}
  larger countries have lower interst rates stock returns traded
  nontraded sector Currency Union

\begin{equation}
  r^{f \ast} + \Delta \mathbb{E} s^{f, h \ast} - r^{h \ast}
  =g\left(\theta^h - \theta^f\right) \sigma_N^2.
  \label{eq_FF_UIP}
\end{equation}
\item \citet{Martin2012}
\item \citet{Richmond2019}

$$\lambda_{T}^\ast = -c
\sum_{n} \theta^n x^n- d\sum_{n} \nu^n x^n$$


\item \citet{Readyetal2013} (and Maggiori JMP?)
$$p^{\text{Commodity Country}}=a\lambda_T+b\frac{\text{Finished goods shipped}}{\text{Shipping Capacity}} $$
Commodity producers appreciate in good times and depreciate in bad
times Their exchange rates are positively correlated with the
commodity price and shipping costs.
\item \citet{FarhiGabaix2016} Countries differ in their resilience to
  disaster risk. Conditional on a disaster striking the world economy,
  the currencies of more resilient countries appreciate.
\item \citet{GourinchasGovillotRey2017} Differences in country size and
  risk aversion
\item Della Corte, Riddiough and Sarno (2016)
\item Tran (2013)
\item Powers (2015)
\item Wiriadinata (2020)
\end{enumerate}

\subsection{Evidence}

[Puzzle 1: microfoundation which is it]

If there is indeed something to this story, it changes how we think
about central issues in macro.


\section{The Interest Differential Puzzle}

[Puzzle 2] [MATH?]

\section{Currency Risk and the allocation of capital across countries}
There are large differences in capital accumulation across countries:

Hassan, Mertens and Zhang (2016)
\begin{equation}\label{eq_link_k_r}
  k^{f\ast}-k^{h\ast} = \frac{\gamma}{\tau(\gamma-1)^2}\left(r^{h \ast} - \Delta \mathbb{E} s^{f, h \ast} - r^{f \ast}\right).
\end{equation}
Implication 1
\begin{enumerate}
\item Lucas (1988)
\end{enumerate}
Implication 2
\begin{enumerate}
\item \citet{CaselliFeyrer2007}
\item \citet{Monge-Naranjo2019}
\end{enumerate}
Implication 3: corporate balance sheets
\begin{enumerate}
\item \citet{Richers2019}
\item di Giovanni, Kalemli-Ozcan, Ulu, Baskaya (2019)
\item \citet{DavidHenriksenSimonovska2014}
\end{enumerate} [picture]
\section{Currency Risk and Exchange Rate Regimes}
Q: Are there other papers that argue policies can manipulate risk
premia?
\begin{equation*}
  p^m = p^{m \ast} + (1 - \theta^m) \zeta (p^{t \ast} - p^{m \ast}).
\end{equation*}
\section{Current Accounts and the Reserve Currency Paradox}
[Puzzle 3: Reserve Currency Paradox] [MATH*]

\section{Dynamics}

HABITS -Theory Verdelhan (2010) Heyerdahl-Larsen (2014) Stathopoulos
(2012) LRR -Theory Colacito and Croce (2011,2013) Colacito, Croce, Ho
and Howard (2018) Bansal Shaliastovich (2013) DISASTERS - Theory
Gourio, Siemer, Verdelhan (2013) Guo (2010) Du (2013) Martin (2013)
Farhi and Gabaix DISASTERS - Empirics


\subsection{Quantitative literature}
\begin{enumerate}
\item \citet{ColacitoCroce2011}
\item \citet{GourioSiemerVerdelhan2011}
\item \citet{ColacitoCroceHoHoward2018}
\end{enumerate}
\subsection{Theories written for FPP}
\begin{enumerate}
    \item \citet{Heyerdahl-Larsen2011}
    \item \citet{Stathopoulos2017}
    \item \citet{Verdelhan2010}
    \item \citet{Backusetal2001}
    \item \citet{BansalShaliastovich2010}
    \item \citet{GabaixMaggiori2015}
\end{enumerate}

\subsection{Measurement of risk}
Maybe measurement of risk - implied vol - commercial indices -
newspapers - conference call transcripts

\subsection{Global financial cycle}

\section{Capital flows and the allocation puzzle}
Capital flows run counter to neo-classical model:
\begin{enumerate}
\item \citet{GourinchasJeanne2013}
\item[-] Current explanations:
\item Differences in financial development (Ju and Wei, 2006;
  Caballero, Farhi and Gouinchas, 2008; Mendoza, Quadrini and
  Rios-Rull, 2009)
\item Differences in factor utilization (Jin, 2012)
\item Differences in contracting frictions (Aguiar and Amador, 2010)
\end{enumerate}


\section{Unsorted}
\begin{enumerate}
\item Gourinchas and Rey (2007)
\item Govillot, Rey and Gourinchas (2010)
\item[-] Implications for equilibrium portfolios and global imbalances
\item Miranda-Agrippino and Rey (2020)
\item[-] Global financial cycles
\item Boccola and Lorenzoni (forthcoming)
\item[-] Debt crises
\item Du and Schreger
\item[-] Corporate balance sheets
\item[-] Special role for the dollar
\item Hassan, Mertens and Zhang (2020)
\item Burnside, Eichenbaum, Kleshchelski and Rebelo (2011)
\end{enumerate}



\subsection{Sidebars and Margin Notes}
% Margin Note
\begin{marginnote}[]
  \entry{Term A}{definition} \entry{Term B}{definition} \entry{Term
    C}{defintion}
\end{marginnote}

\begin{textbox}[h]\section{SIDEBARS}
  Sidebar text goes here.
  \subsection{Sidebar Second-Level Heading}
  More text goes here.\subsubsection{Sidebar third-level heading} Text
  goes here.\end{textbox}


% Summary Points
\begin{summary}[SUMMARY POINTS]
  \begin{enumerate}
  \item Summary point 1. These should be full sentences.
  \item Summary point 2. These should be full sentences.
  \item Summary point 3. These should be full sentences.
  \item Summary point 4. These should be full sentences.
  \end{enumerate}
\end{summary}

% Future Issues
\begin{issues}[FUTURE ISSUES]
  \begin{enumerate}
  \item Future issue 1. These should be full sentences.
  \item Future issue 2. These should be full sentences.
  \item Future issue 3. These should be full sentences.
  \item Future issue 4. These should be full sentences.
  \end{enumerate}
\end{issues}

\newpage

\setstretch{1}

\bibliographystyle{chicago} \bibliography{ARE}

\clearpage


\appendix

\begin{center}
  {\Huge\bf Appendix}\\
  {\large\bf -For online publication only-}
\end{center}

\section{Reduced-Form Results \label{Appendix_ReducedFormResults}}

The country $n$ risk-free bond pays off $P^n$ units of the traded good at maturity. We derive the value of the risk-free bond, $V^n$, by applying the asset pricing equation to the bond payoff: 
\begin{equation*}
  V^n = \mathbb{E}\left[M_{T} P^n
  \right],
\end{equation*}
where $M_{T}$ denotes the stochastic discount factor. The country $n$ risk-free rate (in levels), $R^n$, is the inverse of the price of the risk-free bond:
\begin{equation*}
  R^n = \frac{1}{V^n}.
\end{equation*}
Putting the previous two equations together yields the following relationship:
\begin{equation*}
  \mathbb{E}\left[ M_{T} P^n \right] R^n = 1.
\end{equation*}
As a result, the risk-free rates of countries $f$ and $h$ are related as follows:
\begin{equation*}
  \mathbb{E}\left[M_{T} P^f \right] R^f 
  = \mathbb{E}\left[M_{T} P^h \right] R^h = 1
\end{equation*} 
If the stochastic discount factor and prices are log-normal, we can perform the following calculations:
\begin{align*}
  & \mathbb{E}\left[M_{T} P^f \right] R^f
    = \mathbb{E}\left[M_{T} P^h \right] R^h \\
  \Leftrightarrow\quad
  & \mathbb{E}\left[\exp\left[ m_T + p^f + r^f \right]\right]
    = \mathbb{E}\left[\exp\left[ m_T + p^h + r^h \right]\right] \\
  \Leftrightarrow\quad
  & \mathbb{E}\left[m_T + p^f\right] + \frac{1}{2}\text{var}\left(m_T\right) +      \frac{1}{2}\text{var}\left(p^f\right) + \text{cov}\left(m_{T}, p^f\right) + r^f \\
  & = \mathbb{E}\left[m_{T}+ p^h\right] + \frac{1}{2}\text{var}\left(m_{T}\right) + \frac{1}{2}\text{var}\left(p^h\right) + \text{cov}\left(m_T, p^h\right) + r^h,
\end{align*}
We cancel out $\text{var}\left( m_T \right)$ from both sides of the previous equation.
\begin{align*}
  & \mathbb{E}\left[p^f\right]+\frac{1}{2}\text{var}\left(p^f\right) + \text{cov}\left(m_T, p^f\right) + r^f = \mathbb{E}\left[p^h\right]+\frac{1}{2}\text{var}\left(p^h\right)+\text{cov}\left(m_T,p^h\right) + r^h
  \\      \Leftrightarrow \quad
  & r^f+\mathbb{E}\left[p^f-p^h\right]+\frac{1}{2}\text{var}\left(p^f\right)-\frac{1}{2}\text{var}\left(p^h\right)-r^h = -\text{cov}\left(m_T,p^f-p^h\right)\\   
  \Leftrightarrow\quad
  &r^f+\log\left(\mathbb{E}\left[P^f\right]/\mathbb{E}\left[P^h\right]\right)-r^h = -\text{cov}\left(m_T,p^f-p^h\right)
\end{align*}
We define
$\Delta\mathbb{E}\left[s^{f,h}\right]
=\log\left(\mathbb{E}\left[P^f\right]/\mathbb{E}\left[P^h\right]\right).$
With this definition:
\begin{equation*} 
  r^f+\Delta\mathbb{E}\left[s^{f,h}\right]-r^h = -\text{cov}\left(m_T,p^f-p^h\right). 
\end{equation*}







no arbitrage%
\[
R^{f}=\frac{1}{E\left[ M^{f}\right] }
\]

log normality%
\begin{eqnarray*}
r^{f} &=&-\log \left( E\left[ M^{f}\right] \right)  \\
&=&-\left( E\left[ m^{f}\right] +\frac{1}{2}var\left[ m^{f}\right] \right) 
\end{eqnarray*}%
\begin{eqnarray*}
r^{f}-r^{h} &=&-\left( E\left[ m^{f}\right] +\frac{1}{2}var\left[ m^{f}%
\right] \right) +\left( E\left[ m^{h}\right] +\frac{1}{2}var\left[ m^{h}%
\right] \right)  \\
r^{f}-r^{h}+E\left[ m^{f}-m^{h}\right]  &=&\frac{1}{2}\left( var\left[ m^{h}%
\right] -var\left[ m^{f}\right] \right) 
\end{eqnarray*}

??? assumption, complete markets

\begin{eqnarray*}
r^{f}-r^{h}+\Delta s^{f,h} &=&\frac{1}{2}\left( var\left[ m^{h}\right] -var%
\left[ m^{f}\right] \right)  \\
0.08 &=&\left( var\left[ m^{h}\right] -var\left[ m^{f}\right] \right) 
\end{eqnarray*}%
\bigskip 
    

\end{document}